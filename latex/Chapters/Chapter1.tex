% Chapter Template

\chapter{Introduction} % Main chapter title

\label{Chapter1} % Change X to a consecutive number; for referencing this chapter elsewhere, use \ref{ChapterX}

%----------------------------------------------------------------------------------------
%	SECTION 1
%----------------------------------------------------------------------------------------
Theory of option pricing dates back to Louis Bachelier with his PhD thesis "The Theory of Speculation" in 1900, but it was much later that option theory gained significant attention. In 1973 the world's first option exchange in Chicago opened, and in the same year Fisher Black and Myron Scholes came out with the first analytical formula for the European call option \parencite{B-S-Paper}, this revolutionized market practice and option pricing theory. The idea of replication was born and the financial derivatives could now be priced by rational pricing. \\

The Black-Scholes model for European options is still used today, but the analytical framework cannot handle more complex products such as American put options. Since that time we have seen an increasing complexity of financial products, where big investment- and banks have increased their need for financial engineers to handle the derivative books and price derivative products. With the complexity a lot of challenges have risen in this field, where a great understanding of the products is required to handle big derivative books. Most of the existing derivatives do not have a closed form solution, so numerical methods are used to approximate the price function. To emphasize the risk of derivatives without great understanding the successful trader Warren Buffett says derivatives is "Financial weapons of mass destruction" (p. 15 \parencite{Buffett02}), but on the other hand Warren Buffett has derivatives in his portfolio. A recent example of bad management of a derivative book is AIG, where AIG needed a bailout by the US government under the recent financial crisis \parencite{McDonaldRobert2015}. To sum up derivatives give the trader more options either to utilize arbitrage, speculate or hedge, but without care or knowledge about a derivative book the outcome can be disastrous for the owner.\\

What is the fair price for the right to buy or sell certain derivative for some predetermined exercise dates? And can the existing classical methods be extended by using deep learning? The thesis try to answer these questions by analytical and numerically methods in the Black-Scholes model. The pricing function for the derivatives is essential for handling a derivative book. The aim is to give an accurate and fast method for arbitrage free pricing with neural networks, where classical methods will be used for comparision. The focus in the thesis will be on financial equity contingent claims, where the prime example will be american equity put options with one or two underlying stocks.\\

This thesis starts presenting the basic theory on arbitrage theory to introduce the basic modeling framework for contingent claims. With this introduction the goal is to explore the applications and numerical procedures springing from the underlying theory. The classical methods in chapter \ref{Chapter3} such as the binomial lattice approach and least square monte carlo methods (LSM) will be applied to univariate and bivariate contingent claims, where the presentation of the classical methods are threefold. The methods give a benchmark for the neural network approaches for option pricing in later chapters. Specially the lattice approach gives strong intuition how the optimal stopping problem can be solved, where the LSM gives a framework that is easily extended to methods for pricing multivariate contingent claims. The lattice approach makes it easy to compare values with closed form solution for European options, where some special cases for exotic European option will be presented. \\

Deep learning theory is important for building a sound and high quality model for pricing options with neural networks, hence the basic machine learning and deep learning theory will be presented before the main chapter. Chapter \ref{Chapter5} is the main chapter, where we try to price univariate and bivariate claims with neural networks. The aim is to look for methods which have high accuracy and fast computation time. Chapter \ref{Chapter6} gather all the methods presented and compare them numerically. Chapter \ref{Chapter6} contains also a discussion on the underlying Black-Scholes theory. Finally in chapter \ref{Chapter7}, we conclude on our findings and suggest what can be further investigated.


