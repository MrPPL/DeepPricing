% Chapter Template

\chapter{Introduction} % Main chapter title

\label{Chapter1} % Change X to a consecutive number; for referencing this chapter elsewhere, use \ref{ChapterX}

%----------------------------------------------------------------------------------------
%	SECTION 1
%----------------------------------------------------------------------------------------
In recent years we have seen an increasing complexity of financial products, where big investment- and banks use a lot of money on financial engineerers in creating new innovative products. With the complexity a lot of challenges has risen in this field. Nevertheless the products can help to mitigate risk and leverage your portfolio. A recent example from the financial crisis in 2007 where credit default swap (CDS) almost led to AIGs bailout. A CDS is a derivative, where you insure your risk of losing money on some financial product. The strategy of writing CDS seemed like a good business for AIG as long there was a bull market, because they got good feeds for insuring credit. The CDS was the main reasions that AIG needed a bailout by the US government under the recent financial crisis. In hindsight they wrote to many CDS, hence AIG was too exposed for risk. A great understanding in the financial derivatives is important to understand your risks and ulitmately migate the damage of financial turmoil as Warren Buffett says derivatives is "Financial weapons of mass destruction" (page 15 \parencite{Buffett02}). Eventhough Buffett is critical against derivatives he acknowlegde the usage of derivatives, because he owns derivatives in his portfolio. Derivatives gives the trader more options either to utilize arbitrage, speculate or hedge, but without care or knowledge about your book of derivative the outcome can be disastrous  \parencite{Buffett08}.
\\
\\
The focus is on financial derivatives, where the prime examples will be plain vanilla stock options. We will start with the most basic derivatives European options and move toward more complex products like American options. The European option will be the reference point for our different nummerical approaches to American options, because the European option has a closed form solution (see proposition \ref{BS-price-EuroCall}). When moving into more complex derivatives as American options the Black Scholes analytical framework breaks down, and this calls for numerical methods. We will take different numerical approaches for pricing and hedging, where the ultimate goal is to use machine learning for pricing and hedging.
