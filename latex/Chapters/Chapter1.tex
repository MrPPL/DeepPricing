% Chapter Template

\chapter{Introduction} % Main chapter title

\label{Chapter1} % Change X to a consecutive number; for referencing this chapter elsewhere, use \ref{ChapterX}

%----------------------------------------------------------------------------------------
%	SECTION 1
%----------------------------------------------------------------------------------------
In recent years we have seen an increasing complexity of financial products, where big investment- and banks use a lot of money on financial engineerers in creating new innovative products. With the complexity a lot of challenges has risen in this field. Nevertheless the products can help to risk neutralize your risks. A example would be credit default swap (CDS), where you insure your risk of losing money. On the other hand the CDS was one of the main reasions that AIG needed to be safed by the US government under the recent financial crisis. In hindsight they insured to many with CDS, hence AIG was too exposed when the financial crisis in 2007 hit. A great understanding in the financial derivatives is important to understand your risks.
\parencite{Zucchi}\\
\\
This thesis will focus on financial derivatives, and take different approaches for pricing and hedging. We will start with the most basic derivatives European options and move toward more complex products. The European option will be the reference point for our different approaches, which will ultimately lead to pricing and hedging strategies for other derivatives. European option will be the reference point, because we have an analytic formula (The Black Scholes Formula) for the price. However when moving into other derivatives as American options the Black Scholes analytical framework breaks down, and this calls for numerical methods. In this thesis we will test deep hedging and other numerical methods.