% Chapter Template

\chapter{Conclusion and Further Investigation} % Main chapter title

\label{Chapter7} % Change X to a consecutive number; for referencing this chapter elsewhere, use \ref{ChapterX}

%----------------------------------------------------------------------------------------
%	SECTION 1
%----------------------------------------------------------------------------------------

\section{Conclusion}
We have seen different methods for pricing equity options within the Black-Scholes world. We have specifically focused on the European call, exotic European, American put and American put minimum on two stocks options. The closed form solutions, the classical binomial lattice model and LSM were provided for investigating the potential of using deep learning in option pricing theory. \\

The numerical investigation showed that MLP I and MLP II pricing methods were inferior in accuracy for univariate and bivariate contingent claims, but the MLP II had an enhanced speed compared to the classical methods after training. The results for the MLP I were somewhat discouraging, but the theory tells us that we can improve our training algorithm to perform on the same level as LSM. The CRR model was extended to the BEG method with two underlying stocks, which we saw provided a good approximation for European bivariate contingent claims. The BEG method has its limitations in terms of pricing a basket option with many underlying stocks, which can be solved with LSM or MLP I.\\

To sum up, the MLP I needs hyperparameter tuning, but the underlying theory and idea gives hope for finding a better model for high dimensional data than the classical LSM. The BEG has its limitation in high dimensions. Hence, LSM is preferred for higher dimensions. The MLP II showed better performance than for polynomial regression at supervised learning. Furthermore, the MLP II can be beneficial in terms of computational speed, but has lower precision for univariate contingent claims than the LSM, compared to our CRR benchmark price.\\

The classical methods approximate the true price in the Black-Scholes model well. The deep learning methods lack the accuracy of the classical methods in the Black-Scholes model, hence, there is still some work to be done for pricing derivatives with deep learning. 

%-----------------------------------
%	SECTION 2
%-----------------------------------
\section{Further Investigation}
The code library is written in Python, which could be optimized in terms of computational speed with e.g. Julia or C++. The advantage of Python is the libraries for machine learning and especially deep learning. The code could also be more generalized, not only to handle specific option contracts by using the object oriented programming paradigm instead of procedural programming approach. The code was run on my CPU, but the speed could also be improved by using the GPU.\\

Hyperparameter tuning for the MLP would be interesting to investigate further with e.g. the more advanced method; Bayesian hyperparameter tuning. The article from \parencite{liaw2018tune} looks at how hyperparameter tuning can be easy implemented, which is worth looking at. The data sets could have been bigger or the labels could have been generated with a different method. Another interesting aspect to investigate is calculating the derivatives of the pricing function, which produces the risks for the derivative books. It looks like MLP models are superior for calculating risks in terms of speed to the classical methods, hence, there is potentially an additional gain when using deep learning.\\


