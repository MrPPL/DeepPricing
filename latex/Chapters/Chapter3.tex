% Chapter Template

\chapter{Classical numerical results} % Main chapter title

\label{Chapter3} % Change X to a consecutive number; for referencing this chapter elsewhere, use \ref{ChapterX}


%----------------------------------------------------------------------------------------
%	SECTION 1
%----------------------------------------------------------------------------------------

\section{Binomial Pricing model}
The first numerical example will be the binomial pricing model. We will not go into mathematical details, because we already shown all the concepts and mathmatics needed for continuous time in \ref{Chapter2}. The central concepts arbitrage and completeness from continuous time also work in the discrete time setup, but the mathmatics is simpler. It can actually be shown that if you choose your parameters in this model to follow the mean and variance in the continuous time framework then the discrete time model will converge to the continuous time model. Many argue that the simpler mathematics in this model makes the binomial model more instructive and clear from an economist viewpoint. Besides being easier to understand for non-mathematician is also work nicely with other options than the European options like American options. The classical Black-Scholes model was an analytical result about European options, hence  



The model gives a instructive way of thinking about arbitrage and hedging. 

 Firstly we choose to include the binomial pricing, because it gives a intuition and pure economic reasoning about options pricing without many technical details. Secondly it serves as a benchmark for the other algorithm's, where we can with the model both price European - and American options. This approach differs from the others, because we assume the stock price moves at discrete time instead of continuous time. This approach gives a simplified model in terms of the mathematics and highlights the essential concepts in option pricing theory arbitrage and hedging. Evenlythough our starting point is a binomial process for the development of the stock, it can be shown that in the limit the process for the stock will have a lognormal distribution. Hence the binomial pricing model will be equivalent with the continuos time analytical pricing model derived by Fischer Black and Myron Scholes in the limit \parencite{binomial-Paper}
.

The classical paper \parencite{binomial-Paper} which introduce this binomial model approach to option pricing came after the Black-Scholes model described in section \ref{Chapter2} \parencite{B-S-Paper}. The 



%-----------------------------------
%	SUBSECTION 1
%-----------------------------------
\subsection{Arbitrage and hedging in binomial model}
The important feature of the binomial model is the instructive approach to arbitrage and hedging, without repeating Chapter \ref{Chapter2} will the focus be !!!!!!!!!!!!!!!!!!!

In the binomail model the stock price can either move up or down in each discrete timepoint between !!!!!!!!!!!! and maturiry. For simplicity will we consider a model with one periods after !!!initialization!!!!. The risk neutral formula is the same for the binomial model, except now the expectation is taking wrt. to the counting measure.

\begin{align*}
ss
\end{align*}

The arbitrage price of a option will follow the martingale measure Q, 


%-----------------------------------
%	SUBSECTION 2
%-----------------------------------

\subsection{Numerical results}

%----------------------------------------------------------------------------------------
%	SECTION 2
%----------------------------------------------------------------------------------------

\section{Least Square Monte Carlo Method}


\section{Comparision}

