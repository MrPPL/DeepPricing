% Appendix Template

\chapter{Mathematical results and definitions} % Main appendix title

\label{AppendixB} % Change X to a consecutive letter; for referencing this appendix elsewhere, use \ref{AppendixX}

\begin{theorem}\label{Ito}
\textbf{Itô's formula multidimensional} Let the n-dimensional process X have dynamics given by:
\begin{align}
dX(t)=\mu(t)dt+\sigma(t)dW(t)
\end{align}
Then the process $f(t,X(t))$ has stochastic differential given by:
\begin{equation}
\begin{split}
df(t,X(t))=\frac{\partial f(t,X(t))}{\partial t}  dt + \sum_{i=1}^{n} \frac{\partial f(t,X(t))}{\partial x_i}  dX_i(t) + \frac{1}{2} \sum_{i,j=1}^{n} \frac{\partial^2 f(t,X(t))}{\partial x_i \partial x_j}  dX_i(t)dX_j(t)  
\end{split}
\end{equation}
Note: \[ dW_i \cdot dW_j= \begin{cases} 
      \rho_{ij}dt & \textbf{For correlated Weiner processes} \\
      0 & \textbf{For independent Weiner processes} \\
   \end{cases}
\]
(see page 58-60 \parencite{finKont})
\end{theorem}

\begin{theorem}\label{Girsanov}
\textbf{The Girsanov Theorem} 
Assume the probability space $(\Omega, \mathcal{F}, P, \mathcal{F}_t^{W^p})$ and let the Girsanov kernel $\phi$ be any d-dimensional adapted column vector process. Choose a fixed T and define the process L on $[0,T]$ by:\\
$dL_t=\phi(t)^T\cdot L_t d\bar{W}_t^P$\\
$L_0=1$.\\
Assume that $E^P[L_T]=1$ and define the new probability measure $Q$ on $\mathcal{F}_T$ by:\\
$$L_T=\frac{dQ}{dP} \quad on \ \mathcal{F}_T$$
Then
\begin{align}
d\bar{W}(t)=\phi(t)dt + dW(t)
\end{align}
Where $W(t)$ is the Q-Wiener process and $\bar{W}(t)$ is the P-Wiener process
(see page 164 \parencite{finKont})
\end{theorem}

\theoremstyle{definition}
\begin{definition}{Stopping time in continuous time: }\label{stopTime}
A nonnegative random variable $\tau$ is called a stopping time w.r.t. the filtration $\mathcal{F}$ if it satisfies the condition:
\begin{align}
\{\tau \leq t \} \in \mathcal{F}_t \quad \forall t \geq 0
\end{align}
(see page 329 \parencite{finKont})
\end{definition}

\theoremstyle{definition}
\begin{definition}{Orthogonal vectors: }\label{OrthogonalVec}
Two vectors $\overrightarrow{a}$ and $\overrightarrow{b}$ are orthogonal, if their dot product is 0:
$$\overrightarrow{a} \cdot \overrightarrow{b} = 0$$
We will use the notation:
\begin{equation}
\begin{split}
\overrightarrow{a} \bot \overrightarrow{b}
\end{split}
\end{equation}

\end{definition}