% Appendix A

\chapter{Mathematical results and definitions} % Main appendix title

\label{AppendixA} % Change X to a consecutive letter; for referencing this appendix elsewhere, use \ref{AppendixX}

\begin{theorem}\label{Ito}
\textbf{Itô's formula multidimensional} Let the n-dimensional process X have dynamics given by:
\begin{align}
dX(t)=\mu(t)dt+\sigma(t)dW(t)
\end{align}
Then the process $f(t,X(t))$ has stochastic differential given by:
\begin{equation}
\begin{split}
df(t,X(t))=\frac{\partial f(t,X(t))}{\partial t}  dt + \sum_{i=1}^{n} \frac{\partial f(t,X(t))}{\partial x_i}  dX_i(t) + \frac{1}{2} \sum_{i,j=1}^{n} \frac{\partial^2 f(t,X(t))}{\partial x_i \partial x_j}  dX_i(t)dX_j(t)  
\end{split}
\end{equation}
Note: \[ dW_i(t) \cdot dW_j(t)= \begin{cases} 
      \rho_{ij}dt & \textit{For correlated Weiner processes} \\
      0 & \textit{For independent Weiner processes} \\
   \end{cases}
\]
(page 58-60 \parencite{finKont})
\end{theorem}

\begin{theorem}\label{MRT}
\textbf{The Martingale Representation Theorem} 
Let W be a k-dimensional Wiener process, and assume that the filtration is defined by:
$$\mathcal{F}_t=\mathcal{F}_t^W \quad t\in [0,T]$$
Let M be any $\mathcal{F}_t$-adapted martingale. Then there exist uniquely determined $\mathcal{F}_t$-adapted processes $h_1, \ldots, h_k$ such that M has the representation
$$M(t)=M(0) + \sum_{i=1}^{k} \int_{0}^{t} h_i(s)dW_i(s) \quad t \in [0,T]$$
if the martingale M is square integrable, then $h_1, \ldots, h_k$ are in $L^2$ (page 161 \parencite{finKont}).
\end{theorem}

\begin{theorem}\label{Girsanov}
\textbf{The Girsanov Theorem} 
Assume the probability space $(\Omega, \mathcal{F}, P, (\mathcal{F}_t^{\bar{W}})_{t \in [0,T]})$ and let the Girsanov kernel $\phi$ be any d-dimensional adapted column vector process. Choose a fixed T and define the process L on $[0,T]$ by:\\
$dL_t=\phi(t)^T\cdot L_t d\bar{W}_t^P$\\
$L_0=1$.\\
Assume that $E^P[L_T]=1$ and define the new probability measure $Q$ on $\mathcal{F}_T$ by:\\
$$L_T=\frac{dQ}{dP} \quad on \ \mathcal{F}_T$$
Then
\begin{align}
d\bar{W}(t)=\phi(t)dt + dW(t)
\end{align}
Where $W(t)$ is the Q-Wiener process and $\bar{W}(t)$ is the P-Wiener process
(page 164 \parencite{finKont})
\end{theorem}

\begin{theorem}\label{ConverseGirsanov}
\textbf{The Converse of the Girsanov Theorem} Let $\bar{W}$ be k-dimensional standard P-Wiener process (i.e. zero drift and unit variance independent components) on $(\Omega,\mathcal{F}, P, (\mathcal{F}_t^{\bar{W}})_{t \in [0,T]})$. Assume that there exits a probability measure $Q$ such that $Q<<P$ on $\mathcal{F}_T^{\bar{W}})$. Then there exists an adapted process $\phi$ such that the likelihood process L has dynamics
\begin{align*}
dL(t)=L(t)\phi^T(t)d\bar{W}(t)\\
L(0)=1
\end{align*}
(page 168 \parencite{finKont})
\end{theorem}

\theoremstyle{definition}
\begin{definition}{\textbf{Stopping time}:}\label{StoppingTime}
A random variable $\tau:\Omega \to [0,\infty]$ is called a Markov time if 
$$(\tau \leq t)\in \mathcal{F}_{t} \quad \forall t \geq 0 $$
A Markov time is called a stopping time if $\tau<\infty$ P-a.s.
(p. 27 \parencite{Shiryaev06})
\end{definition}

\theoremstyle{definition}
\begin{definition}{Orthogonal vectors: }\label{OrthogonalVec}
Two vectors $\overrightarrow{a}$ and $\overrightarrow{b}$ are orthogonal, if their dot product is 0:
$$\overrightarrow{a} \cdot \overrightarrow{b} = 0$$
We will use the notation:
\begin{equation}
\begin{split}
\overrightarrow{a} \bot \overrightarrow{b}
\end{split}
\end{equation}
\end{definition}

\theoremstyle{definition}
\begin{definition}{\textbf{Snell envelope}}\label{snellEnvelope}
Consider a fixed process Y.
\begin{enumerate}
\item[•] We say that a process X dominates the process Y if $X_n \geq Y_n \ P-a.s. \ \forall n$.
\item[•] Assuming that $E[Y_n] < \infty  \ \forall n \leq T$, the Snell envelope S, of the process Y is defined as the smallest supermartingale dominating Y. More precisely: S is a supermartingale dominating Y, and if D is another supermartingale dominating Y, then $S_n\leq D_n \ P-a.s. \ \forall n$.
\end{enumerate}
(page 380 \parencite{Bjork19})
\end{definition}

\begin{theorem}\label{SnellEnvelopeTheorem}
\textbf{The Snell Envelope Theorem} The optimal value process V is the Snell envelope of the gain process G
\begin{proof}
see page 381  \parencite{Bjork19}.
\end{proof}
(page 381 \parencite{Bjork19})
\end{theorem}
